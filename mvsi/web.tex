\documentclass[ 12pt]{article}

\usepackage[most]{tcolorbox}

\tcbset{
    frame code={},
    center title,
    left=0pt,
    right=0pt,
    top=0pt,
    bottom=0pt,
    colback=pink!70,
    colframe=white,
    width=\dimexpr\textwidth\relax,
    enlarge left by=0mm,
    boxsep=5pt,
    arc=0pt,outer arc=0pt,
  }


\usepackage{times,vmargin,url}
\setmarginsrb{3cm}{1cm}{2cm}{1cm}{1cm}{1cm}{1cm}{1cm}
%\setpapersize{A4}


\usepackage{color,time}
%\pagecolor{yellow}
%\pagecolor{cyan}
\pagecolor{white}
\title{Modélisation et vérification des systèmes informatiques\\
Cours  pour la formation d'ingénieurs par apprentissage}


\newcommand{\keywords}[1]{\textbf{Keywords:} #1} %\\ \clearpage %\tableofcontents\clearpage}
%

\author{Dominique M\'ery\\
LORIA \& Telecom Nancy\\ Universit\'e de Lorraine\\ \url{dominique.mery@loria.fr}}

\date{\today}

\usepackage{xcolor}
%\usepackage{b2latex}

%\IEEEoverridecommandlockouts
% The preceding line is only needed to identify funding in the first footnote. If that is unneeded, please comment it out.
\usepackage{cite,time}
\usepackage{amsmath,amssymb,amsfonts}
\usepackage{algorithmic}
\usepackage{graphicx}
\usepackage{textcomp}
\usepackage{xcolor,time,url,nbcode,version}
\usepackage{wrapfig}
%\usepackage{b2latex}
\usepackage{caption}
\usepackage{bm}
\usepackage{float}
\usepackage{enumitem}
\usepackage{listings}
\usepackage{bussproofs}
\usepackage{wrapfig,setspace}
%\usepackage{lstcoq}
\usepackage{xspace}
\usepackage{hyperref}

\def\BibTeX{{\rm B\kern-.05em{\sc i\kern-.025em b}\kern-.08em
    T\kern-.1667em\lower.7ex\hbox{E}\kern-.125emX}}
\pagestyle{plain}
\pagenumbering{arabic}
\let\proof\relax
\let\endproof\relax
\let\example\relax
\let\endexample\relax



\usepackage{mathpartir}
\usepackage{amssymb,amsmath,amsthm}
\usepackage{rotating}
\usepackage{blox}
\usepackage{graphicx} 
\usepackage{eventblst}
\usepackage{cite}
\usepackage{dafny}
 

% \usepackage{amssymb,amsmath} problematic with acm, use \usepackage{newtxtext,newtxmath}
% \usepackage{newtxtext,newtxmath}



%%%%%%%%%%%%%%%%%%%%%%%%%%%%%%%%%%%%%%%%%%%%%%%%%%%%%%%%%%%%%%%%%%%%%%%%
% Macros for proof-reading
%%%%%%%%%%%%%%%%%%%%%%%%%%%%%%%%%%%%%%%%%%%%%%%%%%%%%%%%%%%%%%%%%%%%%%%%


\usepackage{xcolor}
\usepackage{ifthen}
\newboolean{showcomments}
\setboolean{showcomments}{true} % toggle to show or hide comments

\ifthenelse{\boolean{showcomments}} { 
    \usepackage{outlines}
    \let\oldoutline\outline
    \def\outline{\oldoutline\color{blue}}

    \usepackage[normalem]{ulem}
    \def \ifempty#1{\def\temp{#1} \ifx\temp\empty }
    \newcommand{\xnote}[3][]{\textcolor{red}{#2}$\rightarrow$ \textcolor{blue}{\textbf{#1} #3}}
    \newcommand{\xtodo}[2][]{\xnote[#1]{}{#2}}

    \newcommand{\xreplace}[3][]{\ifempty {#3} \textcolor{red}{\textbf{#1} \sout{#2}} \else \textcolor{red}{\sout{#2}} $\rightarrow$ \textcolor{blue}{\textbf{#1} "#3"} \fi}
    \newcommand{\xdelete}[2][]{\xreplace[#1]{#2}{}}
    \newcommand{\xadd}[2][]{\xreplace[#1]{}{#2}}
} {
    \newcommand{\xnote}[3][]{}
    \newcommand{\xtodo}[2][]{}
    \newcommand{\xreplace}[3][]{#3}
    \newcommand{\xdelete}[2][]{}
    \newcommand{\xadd}[2][]{#2}
    \def\outline{}
    \def\1{}
    \def\2{}
    \def\3{}
    \def\4{}
}
% % % % % % % % % % % % % % % % % % % % % % % % %

\newcommand{\dnote}[2]{\xnote[DM]{#1}{#2}}
\newcommand{\dtodo}[1]{\xtodo[DM]{#1}}
\newcommand{\dreplace}[2]{\xreplace[DM]{#1}{#2}}
\newcommand{\ddelete}[1]{\xdelete[DM]{#1}}
\newcommand{\dadd}[1]{\xadd[DM]{#1}}


\newcommand{\znote}[2]{\xnote[ZC]{#1}{#2}}
\newcommand{\ztodo}[1]{\xtodo[ZC]{#1}}
\newcommand{\zreplace}[2]{\xreplace[ZC]{#1}{#2}}
\newcommand{\zdelete}[1]{\xdelete[ZC]{#1}}
\newcommand{\zadd}[1]{\xadd[ZC]{#1}}

% % % % % % % % % % % % % % % % % % % % % % % % %

\newcommand{\PB}{{\sc\bf  PB}\;}
\newcommand{\MO}{{\sc\bf  MO}\;}
\newcommand{\GC}{{\sc\bf  GC}\;}
\newcommand{\HP}{{\sc\bf  HP}\;}
\newcommand{\dL}{d${\cal  L}$\;}

\newcommand*\choice[0]{[\!]}
\newcommand{\set}[1]{\{#1\}}
\newcommand{\aloop}[1]{\textbf{do}\  #1  \ \textbf{od}}

\makeatletter
\def\footnoterule{\relax%
  \kern-5pt
  \hbox to \columnwidth{\hfill\vrule width 0.9\columnwidth height 0.4pt\hfill}
  \kern4.6pt}
\makeatother
%ddd -aal\renewcommand{\baselinestretch}{0.90}
\newtheorem{myexample}{Example}[section]
\newtheorem{mytheorem}{Property}
%\newtheorem{definition}{Definition}
%\includeversion{check}
% \excludeversion{check}
% \excludeversion{long}
%%%%%%%%%%%%%%%%%%%%%%%%%%%%%%%%%%%%%%%%%%%%%%%%
%%%%%%%%%%%%%%%%%%%%%%%%%%%%%%%%%%%%%%%%%%%%%%%%
\newtheorem{exemple}{Example}
%\newtheorem{theorem}{Theorem}
%\newcommand{\white}[1]{\textcolor{white}{#1}}

%\newtheorem{comm}{Teaching Point}
%\newenvironment{comm}{\iffalse}{\fi}
%\excludeversion{comm}

\input eb2latex

\pagestyle{plain}
%%%%%%%%%%%%%%%%%%%%%%%%%%%%%%%%%%%%%%%%%%%%%%%%
%%%%%%%%%%%%%%%%%%%%%%%%%%%%%%%%%%%%%%%%%%%%%%%%
\begin{document}


\begin{tcolorbox}



\newcounter{ex}  \setcounter{ex}{1}
\maketitle

Le cours MVSI  est destiné aux élèves apprentis de seconde année  de
Telecom Nancy..
Les objectifs sont de les initier aux techniques de modélisation et
de vérification, en utilisant ds outils comme TLA Toolbox et Frama-c.
Un projet est réalisé  en groupe de deux personnes et appliquent  les
méthodes  et techniques de ce cours.



  \tableofcontents

\section{Documentation et Outils}




\begin{itemize}

  
\item La plateforme TLA Toolbox peut être  obtenue en visitant le site  \href{https://lamport.azurewebsites.net/tla/toolbox.html}{The TLA+ Toolbox}.


  
\end{itemize}


\section{Slides du cours}
\label{sec:course-mcfsi-at}




\subsection{Cours 0
  \href{http://mery54.github.io/teaching/mvsi/lecturesnotes/mvsilecture0.pdf}{CM0}. }




\subsection{Cours 1
  \href{http://mery54.github.io/teaching/mvsi/lecturesnotes/mvsilecture1.pdf}{CM1}. }


\subsection{Cours 2
  \href{http://mery54.github.io/teaching/mvsi/lecturesnotes/mvsilecture2.pdf}{CM2}. }


\subsection{Cours 3
  \href{http://mery54.github.io/teaching/mvsi/lecturesnotes/lectures-app-2024-cm3.pdf}{CM3}. }



\subsection{Cours 4
  \href{http://mery54.github.io/teaching/mvsi/lecturesnotes/lectures-app-2024-cm4.pdf}{CM4}. }




\subsection{Cours 5
  \href{http://mery54.github.io/teaching/mvsi/lecturesnotes/lectures-app-2024-2.pdf}{CM5}. }




\subsection{Cours 6
  \href{http://mery54.github.io/teaching/mvsi/lecturesnotes/lectures-app-2024-3.pdf}{CM6}. }

  

\section{Travaux Dirigés}

  \begin{itemize}
  \item[]   TD 1 
    \href{http://mery54.github.io/teaching/mvsi/lecturesnotes/newtd1.pdf}{Série d'exercices sur la od\'elisation TLA$^+$  (1)}.

       \href{http://mery54.github.io/teaching/mvsi/lecturesnotes/cnewtd1.pdf}{Correction 
       TD1}.

       
  \item[]   TD 2 
    \href{http://mery54.github.io/teaching/mvsi/lecturesnotes/td2.pdf}{Série
      d'exercices sur la mod\'elisation TLA$^+$ (2).}   

          \href{http://mery54.github.io/teaching/mvsi/lecturesnotes/ctd2.pdf}{Correction 
            TD2}.
          
  \item[]   TD 3 
    \href{http://mery54.github.io/teaching/mvsi/lecturesnotes/td3.pdf}{
      Série sur l'annotation, le contrat, la  mod\'elisation et  la v\'erification.}

      \href{http://mery54.github.io/teaching/mvsi/lecturesnotes/ctd3.pdf}{Correction 
       TD3}.

\item[]   TD 4
    \href{http://mery54.github.io/teaching/mvsi/lecturesnotes/td4.pdf}{
      Série sur la mod\'elisation d'algorithmes en PlusCal  (I).}


      \href{http://mery54.github.io/teaching/mvsi/lecturesnotes/ctd4.pdf}{Correction 
       TD4}.


\item[]   TD 5
    \href{http://mery54.github.io/teaching/mvsi/lecturesnotes/td5.pdf}{
      Série sur la mod\'elisation d'algorithmes en PlusCal  (II).}


      \href{http://mery54.github.io/teaching/mvsi/lecturesnotes/ctd5.pdf}{Correction 
       TD5}.

     
\item[]   TD 6
    \href{http://mery54.github.io/teaching/mvsi/lecturesnotes/td6.pdf}{
      Série sur l'utilisation d'un environnement de  v\'erification Frama-c 
      (I).}
    

      \href{http://mery54.github.io/teaching/mvsi/lecturesnotes/ctd6.pdf}{Correction 
       TD6}.
    
     
\item[]   TD 7
    \href{http://mery54.github.io/teaching/mvsi/lecturesnotes/td7.pdf}{
      Série sur l'utilisation d'un environnement de  v\'erification Frama-c 
      (II).}
    

      \href{http://mery54.github.io/teaching/mvsi/lecturesnotes/ctd7.pdf}{Correction 
       TD7}.

     
\item[]   TD 8
    \href{http://mery54.github.io/teaching/mvsi/lecturesnotes/td8.pdf}{
      Série sur l'utilisation d'un environnement de  v\'erification Frama-c 
      (III).}
    

%      \href{http://mery54.github.io/teaching/mvsi/lecturesnotes/ctd8.pdf}{Correction         TD8}.

     

  \end{itemize}

  
  \section{Projet MVSI}
\label{sec:project}

Le  projet est décrit   dans ce     \href{http://mery54.github.io/teaching/mvsi/lecturesnotes/projet2024.pdf}{document}.

Vous devez  constituer des groupes de deux personnes   et choisir un
sujet  différent pour chaque groupe.




  \section{Modèles TLA}
\label{sec:event-b-models}


\subsection{TD1}

 \begin{itemize}
  \item[]   appex1\_1.tla 
    \href{http://mery54.github.io/teaching/mvsi/models/appex1_1.tla}{
      appec1\_1.tla }.
      \item[]   appex1\_2.tla 
    \href{http://mery54.github.io/teaching/mvsi/models/appex1_2.tla}{
      appec1\_2.tla }.
       \item[]   appex1\_3.tla 
    \href{http://mery54.github.io/teaching/mvsi/models/appex1_3.tla}{
      appec1\_3.tla }.
          \item[]   appex1\_4.tla 
    \href{http://mery54.github.io/teaching/mvsi/models/appex1_4.tla}{
      appec1\_4.tla }.
  \end{itemize}

  
\subsection{TD2}
    
 \begin{itemize}
  \item[]   appex2\_1.tla 
    \href{http://mery54.github.io/teaching/mvsi/models/appex2_1.tla}{
      appec2\_1.tla }.
      \item[]   appex2\_2.tla 
    \href{http://mery54.github.io/teaching/mvsi/models/appex2_2.tla}{
      appec2\_2.tla }.
       \item[]   appex2\_3.tla 
    \href{http://mery54.github.io/teaching/mvsi/models/appex2_3.tla}{
      appec2\_3.tla }.
          \item[]   appex2\_4.tla 
    \href{http://mery54.github.io/teaching/mvsi/models/appex2_4.tla}{
      appec2\_4.tla }.
  \end{itemize}

  
\subsection{TD3}
    
\begin{itemize}

  \item[]   appex3\_1.tla 
    \href{http://mery54.github.io/teaching/mvsi/models/appex3_1.tla}{
      appec3\_1ls a.tla }.

    
  \item[]   appex3\_5.tla 
    \href{http://mery54.github.io/teaching/mvsi/models/appex3_5.tla}{
      appec3\_5.tla }.

       \href{http://mery54.github.io/teaching/mvsi/models/appex3_5.pdf}{
      appec3\_5.pdf }.

      \item[]   appex3\_7.tla 
        \href{http://mery54.github.io/teaching/mvsi/models/appex3_7.tla}{
          appec3\_7.tla }.

              \item[]   appex3\_8.tla 
        \href{http://mery54.github.io/teaching/mvsi/models/appex3_8.tla}{
          appec3\_8.tla }.

              \item[]   appex3\_10.tla 
        \href{http://mery54.github.io/teaching/mvsi/models/appex3_10.tla}{
          appec3\_10.tla }.
        
   \end{itemize}


\section{Annales   des examens passés}

\href{http://mery54.github.io/teaching/mvsi/lecturesnotes/mvsi-annales.pdf}{Students
  can obtained former  exams.}.
        
   
\bibliographystyle{plain}

\bibliography{references}
\end{tcolorbox}
\end{document}


