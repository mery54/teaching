\documentclass[ 12pt]{article}
%\documentclass[ 12pt]{book}

\usepackage{times,vmargin,url}
\setmarginsrb{3cm}{1cm}{2cm}{1cm}{1cm}{1cm}{1cm}{1cm}
%\setpapersize{A4}


\usepackage{color,time}
%\pagecolor{yellow}
%\pagecolor{cyan}
\pagecolor{white}
\title{Modelling Software-based Systems}


\newcommand{\keywords}[1]{\textbf{Keywords:} #1} %\\ \clearpage %\tableofcontents\clearpage}
%

\author{Dominique M\'ery\\
LORIA \& Telecom Nancy\\ Universit\'e de Lorraine\\
\url{https://members.loria.fr/Mery}\\ \url{dominique.mery@loria.fr}}

\date{\today}

\usepackage{xcolor}
%\usepackage{b2latex}

%\IEEEoverridecommandlockouts
% The preceding line is only needed to identify funding in the first footnote. If that is unneeded, please comment it out.
\usepackage{cite,time}
\usepackage{amsmath,amssymb,amsfonts}
\usepackage{algorithmic}
\usepackage{graphicx}
\usepackage{textcomp}
\usepackage{xcolor,time,url,nbcode,version}
\usepackage{wrapfig}
%\usepackage{b2latex}
\usepackage{caption}
\usepackage{bm}
\usepackage{float}
\usepackage{enumitem}
\usepackage{listings}
\usepackage{bussproofs}
\usepackage{wrapfig,setspace}
%\usepackage{lstcoq}
\usepackage{xspace}
\usepackage{hyperref}

\def\BibTeX{{\rm B\kern-.05em{\sc i\kern-.025em b}\kern-.08em
    T\kern-.1667em\lower.7ex\hbox{E}\kern-.125emX}}
\pagestyle{plain}
\pagenumbering{arabic}
\let\proof\relax
\let\endproof\relax
\let\example\relax
\let\endexample\relax



\usepackage{mathpartir}
\usepackage{amssymb,amsmath,amsthm}
\usepackage{rotating}
\usepackage{blox}
\usepackage{graphicx} 
\usepackage{eventblst}
\usepackage{cite}
\usepackage{dafny}
 

% \usepackage{amssymb,amsmath} problematic with acm, use \usepackage{newtxtext,newtxmath}
% \usepackage{newtxtext,newtxmath}



%%%%%%%%%%%%%%%%%%%%%%%%%%%%%%%%%%%%%%%%%%%%%%%%%%%%%%%%%%%%%%%%%%%%%%%%
% Macros for proof-reading
%%%%%%%%%%%%%%%%%%%%%%%%%%%%%%%%%%%%%%%%%%%%%%%%%%%%%%%%%%%%%%%%%%%%%%%%


\usepackage{xcolor}
\usepackage{ifthen}
\newboolean{showcomments}
\setboolean{showcomments}{true} % toggle to show or hide comments

\ifthenelse{\boolean{showcomments}} { 
    \usepackage{outlines}
    \let\oldoutline\outline
    \def\outline{\oldoutline\color{blue}}

    \usepackage[normalem]{ulem}
    \def \ifempty#1{\def\temp{#1} \ifx\temp\empty }
    \newcommand{\xnote}[3][]{\textcolor{red}{#2}$\rightarrow$ \textcolor{blue}{\textbf{#1} #3}}
    \newcommand{\xtodo}[2][]{\xnote[#1]{}{#2}}

    \newcommand{\xreplace}[3][]{\ifempty {#3} \textcolor{red}{\textbf{#1} \sout{#2}} \else \textcolor{red}{\sout{#2}} $\rightarrow$ \textcolor{blue}{\textbf{#1} "#3"} \fi}
    \newcommand{\xdelete}[2][]{\xreplace[#1]{#2}{}}
    \newcommand{\xadd}[2][]{\xreplace[#1]{}{#2}}
} {
    \newcommand{\xnote}[3][]{}
    \newcommand{\xtodo}[2][]{}
    \newcommand{\xreplace}[3][]{#3}
    \newcommand{\xdelete}[2][]{}
    \newcommand{\xadd}[2][]{#2}
    \def\outline{}
    \def\1{}
    \def\2{}
    \def\3{}
    \def\4{}
}
% % % % % % % % % % % % % % % % % % % % % % % % %

\newcommand{\dnote}[2]{\xnote[DM]{#1}{#2}}
\newcommand{\dtodo}[1]{\xtodo[DM]{#1}}
\newcommand{\dreplace}[2]{\xreplace[DM]{#1}{#2}}
\newcommand{\ddelete}[1]{\xdelete[DM]{#1}}
\newcommand{\dadd}[1]{\xadd[DM]{#1}}


\newcommand{\znote}[2]{\xnote[ZC]{#1}{#2}}
\newcommand{\ztodo}[1]{\xtodo[ZC]{#1}}
\newcommand{\zreplace}[2]{\xreplace[ZC]{#1}{#2}}
\newcommand{\zdelete}[1]{\xdelete[ZC]{#1}}
\newcommand{\zadd}[1]{\xadd[ZC]{#1}}

% % % % % % % % % % % % % % % % % % % % % % % % %

\newcommand{\PB}{{\sc\bf  PB}\;}
\newcommand{\MO}{{\sc\bf  MO}\;}
\newcommand{\GC}{{\sc\bf  GC}\;}
\newcommand{\HP}{{\sc\bf  HP}\;}
\newcommand{\dL}{d${\cal  L}$\;}

\newcommand*\choice[0]{[\!]}
\newcommand{\set}[1]{\{#1\}}
\newcommand{\aloop}[1]{\textbf{do}\  #1  \ \textbf{od}}

\makeatletter
\def\footnoterule{\relax%
  \kern-5pt
  \hbox to \columnwidth{\hfill\vrule width 0.9\columnwidth height 0.4pt\hfill}
  \kern4.6pt}
\makeatother
%ddd -aal\renewcommand{\baselinestretch}{0.90}
\newtheorem{myexample}{Example}[section]
\newtheorem{mytheorem}{Property}
%\newtheorem{definition}{Definition}
%\includeversion{check}
% \excludeversion{check}
% \excludeversion{long}
%%%%%%%%%%%%%%%%%%%%%%%%%%%%%%%%%%%%%%%%%%%%%%%%
%%%%%%%%%%%%%%%%%%%%%%%%%%%%%%%%%%%%%%%%%%%%%%%%
\newtheorem{exemple}{Example}
%\newtheorem{theorem}{Theorem}
%\newcommand{\white}[1]{\textcolor{white}{#1}}

%\newtheorem{comm}{Teaching Point}
%\newenvironment{comm}{\iffalse}{\fi}
%\excludeversion{comm}

\input eb2latex

\pagestyle{plain}
%%%%%%%%%%%%%%%%%%%%%%%%%%%%%%%%%%%%%%%%%%%%%%%%
%%%%%%%%%%%%%%%%%%%%%%%%%%%%%%%%%%%%%%%%%%%%%%%%
\begin{document}
\newcounter{ex}  \setcounter{ex}{1}
\maketitle

This repository contains course notes, exercises, models and projects
from three courses given as part of master's level training on the
Event-B formal method. It provides access to resources in the form of
pdf files or Rodin archives. The relationship with the Atelier-B
platform is explained.

The table
of contents shows the  summary of three main courses (at Université de 
 Lorraine/University of Lorraine)  based on our
experiment using the  Event-B method:
\begin{itemize}
\item The first course \textbf{MCFSI} is part of the curriculum of the last yer
  students of Telecom Nancy  who are  focusing on software
  engineering.
\item  The second course  \textbf{MsC in Computer Scince} is aimed at students on the IT Masters course
  at the University of Lorraine.
\item  The third course \textbf{DISCONT} is aimed at Telecom Nancy students in their final year who are specialising in software engineering and embedded systems. The course focuses on hybrid systems and hybrid models. The work of the ANR DISCONT project (see 
\href{https://anr.fr/Projet-ANR-17-CE25-0005}{
   website at ANR} or  \href{https://discont.loria.fr}{  website at LORIA} )
\end{itemize}



  The Event-B  method is based on a modelling
 language used to describe state-based models and  safety
 properties of those state-based models.  The originality of  Event-B
 lies in its ability  to
 enable incremental and proof-based modelling of \textit{reactive
   systems}. The  Event-B   language contains both set notations and a
 first-order predicate calculus; it offers the possibility of defining
 models of reactive systems called machines and contexts and includes
  the refinement relationship that allows us to follow an incremental
  development methodology.  



\tableofcontents


\section{Documentation and Tools}




\begin{itemize}

\item[]  The main document on the Event-B  modelling language and
  methodology is the book of Jean-Raymond Abrial \cite{abrial2010}.

  
  
\item[]  The Rodin platform is available at the 
  \href{https://www.event-b.org/install.html}{following 
    link}.


  
\item[]  The list of 
  \href{http://mery54.github.io/teaching/mosos/lecturesnotes/symboles.pdf}{
    symbols } is useful for   typing symbols  which  will appear as 
  mathematical notations.



\item[]  The first chapter on the modelling language Event-B
  \href{http://mery54.github.io/teaching/mosos/lecturesnotes/main-C1.pdf}{chapter
  1} summarizes    details on the foundations and on the applications.


\item[]  The second  chapter on using the modelling language Event-B
  for developing sequential algorithms
  \href{http://mery54.github.io/teaching/mosos/lecturesnotes/main-C2.pdf}{chapter
  2}  with examples.



\item[]  The third  chapter on using the modelling language Event-B
  for verifying contracts for  sequential algorithms
  \href{http://mery54.github.io/teaching/mosos/lecturesnotes/main-fmt.pdf}{chapter
  3}  with examples.


  
\item[]  The chapter \cite{losl-b} (from \cite{losl-eatcs}.)  authored
  by Dominique Cansell and   Dominique Méry and entitled  \textit{The
    Event-B Modelling   Method: Concepts and Case        Studies} has
  benne published from lectures notes given in a Summer School and you can use it for
getting details from  Event-B   see  the  \href{http://mery54.github.io/teaching/mosos/lecturesnotes/BasicEventB.pdf}{following
  link}.

\item[]  Lectures Notes   \textit{The Modelling Language} at the 
  \href{http://mery54.github.io/teaching/mosos/lecturesnotes/main-Poly1.pdf}{following 
    link}.



  
\item[]  Chapter   \textit{Event-B} at the 
  \href{http://mery54.github.io/teaching/mosos/lecturesnotes/Chapter10.pdf}{following     link}.



  
\end{itemize}




\section{Course MCFSI  at Telecom Nancy}
\label{sec:course-mcfsi-at}


\subsection{Slides for the course MCFSI}
\label{sec:slides}



\subsubsection{Lecture 1
  \href{http://mery54.github.io/teaching/mosos/lecturesnotes/mcfsi-lect1.pdf}{The 
    Modelling Language Event-B}. 
}


\subsubsection{Lecture 2
    \href{http://mery54.github.io/teaching/mosos/lecturesnotes/mcfsi-po.pdf}{Proof
      Obligations}.}


  
\subsubsection{Lecture 3
    \href{http://mery54.github.io/teaching/mosos/lecturesnotes/mcfsi-verification.pdf}{Checking
      contracts using Event-B}.}

  

  \subsubsection{Lecture 4
    \href{http://mery54.github.io/teaching/mosos/lecturesnotes/mcfsi-lect2.pdf}{Correctness by Construction with the  
      Modelling Language Event-B using the Refinement}.}
\label{sec:lect-3-hrefhttp:m}


\subsubsection{ Lecture 5 
    \href{http://mery54.github.io/teaching/mosos/lecturesnotes/mcfsi-lect3.pdf}{Access 
      Control }
}



\subsection{Tutorials}

  \begin{itemize}
  \item[]   Tutorial 1
    \href{http://mery54.github.io/teaching/mosos/lecturesnotes/mcfsi-td1.pdf}{Using
      the Event-B modelling language on simple examples}.

   \item[]   Tutorial 2
    \href{http://mery54.github.io/teaching/mosos/lecturesnotes/mcfsi-td2.pdf}{Using
      the Event-B modelling language  for verifying contracts}.



    \item[]   Tutorial 3
    \href{http://mery54.github.io/teaching/mosos/lecturesnotes/mcfsi-td3.pdf}{Using
      the refinement in 
      the Event-B modelling language  for derinving sequential algorithms}.


    
    \item[]   Tutorial 4
    \href{http://mery54.github.io/teaching/mosos/lecturesnotes/mcfsi-td4.pdf}{Using
      the refinement in 
      the Event-B modelling language  for derinving  systems}.


  \end{itemize}

  
  \subsection{Project}
\label{sec:project}

The assessment  of students is based on two works:
\begin{itemize}
\item A written  exam
\item A project which is stated in the following document
  \href{http://mery54.github.io/teaching/mosos/lecturesnotes/telecom-projects2425.pdf}{Projet
    3A IL} 
\end{itemize}

The deadline for the project is  February 1, 2025 and the presentation
is organised on Monday, February 10, 2025. 

  \subsection{Event-B Models}
\label{sec:event-b-models}


The Event-B models related to the tutoriuals are given in the next list: 
% are at the  \href{http://mery54.github.io/teaching/models/}{link for  \textsf{Event-B} }. 

\subsubsection{Event-B Archives for the  lectures}

 \href{http://mery54.github.io/teaching/mosos/lecturesnotes/
   ex-safety.zip}{Archive Rodin  for  explaining differences between
   Event-B invariant and Eveny-B theorem. }


 
 \href{http://mery54.github.io/teaching/mosos/lecturesnotes/
   ex-school.zip}{Archive Rodin  for   the management of school. }

 
 
 \href{http://mery54.github.io/teaching/mosos/lecturesnotes/
   clock-tut0.zip}{Archive Rodin  for  clock modelling. }

  \href{http://mery54.github.io/teaching/mosos/lecturesnotes/factorial-plugin-tutO.zip}{Archive Rodin  for   modelling  the 
   design of the factorial function.}
  


 \subsubsection{Event-B Archives for   the tutorial 1}
\label{sec:event-b-archives}



\href{http://mery54.github.io/teaching/mosos/models/mcfsi1-ex1-tut1.zip}{Archive
  Rodin   mcfsi1-ex1-tut1.zip.}

\href{http://mery54.github.io/teaching/mosos/models/mcfsi1-ex1-tut1.zip}{Archive
  Rodin   mcfsi1-ex2-tut1.zip.}


\href{http://mery54.github.io/teaching/mosos/models/mcfsi1-simple.zip}{Archive 
  Rodin   mcfsi1-simple.zip.}

\href{http://mery54.github.io/teaching/mosos/models/mcfsi1-variant1.zip}{Archive 
  Rodin   mcfsi1-variant1.zip.}

\href{http://mery54.github.io/teaching/mosos/models/mcfsi1-variant2.zip}{Archive 
  Rodin   mcfsi1-variant2.zip.}


\href{http://mery54.github.io/teaching/mosos/models/mcfsi1-summation.zip}{Archive 
  Rodin   mcfsi1-summation.zip.}


\href{http://mery54.github.io/teaching/mosos/models/mcfsi1-ressource-pb1.zip}{Archive 
  Rodin   mcfsi1-ressource-pb1.zip.}


\href{http://mery54.github.io/teaching/mosos/models/mcfsi1-ressource-pb1.zip}{Archive 
  Rodin   mcfsi1-ressource-pb2.zip.}





\href{http://mery54.github.io/teaching/mosos/models/mcfsi1-invariantsafety.zip}{Archive 
  Rodin   mcfsi1-invariantsafety.zip.}



\href{http://mery54.github.io/teaching/mosos/models/mcfsi1-ex8.zip}{Archive 
  Rodin   mcfsi1-ex8.zip.}

\href{http://mery54.github.io/teaching/mosos/models/mcfsi1-ex9.zip}{Archive 
  Rodin   ex8-tut1.zip.}



\href{http://mery54.github.io/teaching/mosos/models/mcfsi1-ex9.zip}{Archive 
  Rodin   mcfsi1-ex9.zip.}



 \subsubsection{Event-B Archives for   the tutorial 2}
\label{sec:event-b-archives}




\href{http://mery54.github.io/teaching/mosos/models/alg-maxtwonumbers.zip}{Archive 
  Rodin  alg-maxtwonumbers.zp.}





\href{http://mery54.github.io/teaching/mosos/models/alg-ex1.zip}{Archive 
  Rodin  alg-ex1.zp.}

\href{http://mery54.github.io/teaching/mosos/models/alg-ex2.zip}{Archive 
  Rodin  alg-ex2.zp.}

\href{http://mery54.github.io/teaching/mosos/models/alg-ex3.zip}{Archive 
  Rodin  alg-ex3.zp.}

\href{http://mery54.github.io/teaching/mosos/models/alg-ex4.zip}{Archive 
  Rodin  alg-ex4.zp.}




\href{http://mery54.github.io/teaching/mosos/models/alg-simple.zip}{Archive 
  Rodin  alg-simple.zip.}





 \subsubsection{Event-B Archives for   the tutorial 3}
\label{sec:event-b-archives}





\href{http://mery54.github.io/teaching/mosos/models/mcfsi3-ex1-plugin.zip}{Archive 
  Rodin  mcfsi3-ex1-plugin.zip.}

\href{http://mery54.github.io/teaching/mosos/models/mcfsi3-ex2.zip}{Archive 
  Rodin  mcfsi3-ex2.zip.}

\href{http://mery54.github.io/teaching/mosos/models/mcfsi3-ex2-plugin.zip}{Archive 
  Rodin  mcfsi3-ex2.zip.}



\href{http://mery54.github.io/teaching/mosos/codes/power2.c}{Fichier 
  power2.c.}

\href{http://mery54.github.io/teaching/mosos/codes/power2.h}{Fichier 
  power2.h.}


\href{http://mery54.github.io/teaching/mosos/models/mcfsi3-ex3.zip}{Archive 
  Rodin  mcfsi3-ex3.zip.}



\href{http://mery54.github.io/teaching/mosos/models/mcfsi3-ex4.zip}{Archive 
  Rodin  mcfsi3-ex4.zip.}



\href{http://mery54.github.io/teaching/mosos/models/mcfsi3-ex5.zip}{Archive 
  Rodin  mcfsi3-ex5.zip.}



 \subsubsection{Event-B Archives for   the tutorial 4}
\label{sec:event-b-archives}





\href{http://mery54.github.io/teaching/mosos/models/mcfsi4-ex1.zip}{Archive 
  Rodin  mcfsi4-ex1.zip.}


\href{http://mery54.github.io/teaching/mosos/models/mcfsi4-ex2.zip}{Archive 
  Rodin  mcfsi4-ex2.zip.}





\href{http://mery54.github.io/teaching/mosos/models/mcfsi4-ex3.zip}{Archive 
  Rodin  mcfsi4-ex3.zip.}




\href{http://mery54.github.io/teaching/mosos/models/mcfsi4-ex4.zip}{Archive 
  Rodin  mcfsi4-ex4.zip.}





 \subsubsection{Event-B Archives}
\label{sec:event-b-archives}






\href{http://mery54.github.io/teaching/mosos/models/mrg1.zip}{Archive 
  Rodin  mrg1.zip.}

\href{http://mery54.github.io/teaching/mosos/models/abacus.zip}{Archive 
  Rodin  abacus.zip.}


\subsection{Past exams  of  the course MCFSI}
\label{sec:past-exams-course}


\href{http://mery54.github.io/teaching/mosos/lecturesnotes/exam2017.pdf}{Exam   2017}


\href{http://mery54.github.io/teaching/mosos/lecturesnotes/exam2018.pdf}{Exam   2018}


\href{http://mery54.github.io/teaching/mosos/lecturesnotes/exam2019.pdf}{Exam   2019}

\href{http://mery54.github.io/teaching/mosos/lecturesnotes/exam2020.pdf}{Exam   2020}

\href{http://mery54.github.io/teaching/mosos/lecturesnotes/exam2021.pdf}{Exam   2021}


\href{http://mery54.github.io/teaching/mosos/lecturesnotes/exam2022.pdf}{Exam   2022}


\href{http://mery54.github.io/teaching/mosos/lecturesnotes/exam2023.pdf}{Exam 
  2023}


\hrulefill






\section{Course  MsC in Computer Science:  Modelling and verifying software-based systems for
  Master in Compurer Science of the University of Lorraine}
\label{sec:course-modell-verify}


\subsection{Slides of the course}
\label{sec:slides-course}

\subsubsection{Lecture 1 
  \href{http://mery54.github.io/teaching/mosos/lecturesnotes/masterillect1-1.pdf}{The 
    Modelling Language Event-B}.}

  
\subsubsection{ Lecture 2 
    \href{http://mery54.github.io/teaching/mosos/lecturesnotes/masterillect1-2.pdf}{Proof 
      Obligations}.}

\subsubsection{Lecture 3 
    \href{http://mery54.github.io/teaching/mosos/lecturesnotes/masterillect2.pdf}{Correctness by Construction with the  
      Modelling Language Event-B using the Refinement}.}

  \subsubsection{ Lecture 4 
    \href{http://mery54.github.io/teaching/mosos/lecturesnotes/masterillect3.pdf}{Access 
      Control }}


\subsubsection{ Lecture 5
    \href{http://mery54.github.io/teaching/mosos/lecturesnotes/masterillect-verification.pdf}{Checking contracts with Event-B}}

      
  
\subsection{Tutorials}

  \begin{itemize}
  \item[]   Tutorial 1
    \href{http://mery54.github.io/teaching/mosos/lecturesnotes/master-tutorial1.pdf}{Using 
      the Event-B modelling language}.

     \item[]   Tutorial 2
    \href{http://mery54.github.io/teaching/mosos/lecturesnotes/master-tutorial2.pdf}{
      Designing and verifying sequential algorithms using the Event-B
      modelling language}.

        \item[]   Tutorial 3
    \href{http://mery54.github.io/teaching/mosos/lecturesnotes/master-tutorial3.pdf}{
      Modelling   systems  using the Event-B}.

   \item[]   Tutorial 4
    \href{http://mery54.github.io/teaching/mosos/lecturesnotes/master-tutorial4.pdf}{
      Using  Event-B  for  verifying sequential  annoptated algorithms}.


  
   \item[]   Tutorial 5
    \href{http://mery54.github.io/teaching/mosos/lecturesnotes/master-tutorial5.pdf}{
      Still refinement \ldots}.

  
    
  \end{itemize}



  \subsection{Event-B Models}
\label{sec:event-b-models}


The Event-B models related to the tutoriuals are given in the next list: 
% are at the  \href{http://mery54.github.io/teaching/models/}{link for  \textsf{Event-B} }. 

\subsubsection{Event-B Archives for the  lectures}

 \href{http://mery54.github.io/teaching/mosos/lecturesnotes/
   ex-safety.zip}{Archive Rodin  for  explaining differences between
   Event-B invariant and Eveny-B theorem. }


 
 \href{http://mery54.github.io/teaching/mosos/lecturesnotes/
   ex-school.zip}{Archive Rodin  for   the management of school. }
  


 \subsubsection{Event-B Archives for   the tutorial 1}
\label{sec:event-b-archives}



\href{http://mery54.github.io/teaching/mosos/models/ex1-tut1.zip}{Archive
  Rodin   ex1-tut1.zip.}

\href{http://mery54.github.io/teaching/mosos/models/ex2-tut1.zip}{Archive 
  Rodin   ex2-tut1.zip.}

\href{http://mery54.github.io/teaching/mosos/models/ex4-tut1.zip}{Archive 
  Rodin   ex4-tut1.zip.}


\href{http://mery54.github.io/teaching/mosos/models/ex51-tut1.zip}{Archive 
  Rodin   ex51-tut1.zip.}

\href{http://mery54.github.io/teaching/mosos/models/ex52-tut1.zip}{Archive 
  Rodin   ex52-tut1.zip.}



\href{http://mery54.github.io/teaching/mosos/models/ex6-tut1.zip}{Archive 
  Rodin   ex6-tut1.zip.}



\href{http://mery54.github.io/teaching/mosos/models/ex7-tut1.zip}{Archive 
  Rodin   ex7-tut1.zip.}


\href{http://mery54.github.io/teaching/mosos/models/ex8-tut1.zip}{Archive 
  Rodin   ex8-tut1.zip.}


\href{http://mery54.github.io/teaching/mosos/models/ex9-tut1.zip}{Archive 
  Rodin   ex9-tut1.zip.}



\href{http://mery54.github.io/teaching/mosos/models/mcfsi1-variant1.zip}{Archive 
  Rodin   mcfsi1-variant1.zip.}

\href{http://mery54.github.io/teaching/mosos/models/mcfsi1-variant2.zip}{Archive 
  Rodin   mcfsi1-variant2.zip.}


\href{http://mery54.github.io/teaching/mosos/models/ex10-1-tut1.zip}{Archive 
  Rodin   ex10-1-tut1.zip.}

\href{http://mery54.github.io/teaching/mosos/models/ex10-2-tut1.zip}{Archive 
  Rodin   ex10-2-tut1.zip.}



 \subsubsection{Event-B Archives for   the tutorial 2}
\label{sec:event-b-archives}

\paragraph{Exercice 1 Tutorial 2}

Nous donnons deux solutions possibles: l'une avec ok et l'autre sans ok.

\href{http://mery54.github.io/teaching/mosos/models/fx1-tut2.zip}{Archive 
  Rodin   fx1-tut2.zip.}


\href{http://mery54.github.io/teaching/mosos/models/fx1-tut2.zip}{Archive 
  Rodin   fx1-tut2bis.zip.}


\paragraph{Exercice 2 Tutorial 2}



\href{http://mery54.github.io/teaching/mosos/models/fx2-tut2.zip}{Archive 
  Rodin   fx2-tut2.zip.}


\paragraph{Exercice 3 Tutorial 2}
\href{http://mery54.github.io/teaching/mosos/models/mcfsi3-ex3.zip}{Archive 
  Rodin  mcfsi3-ex3.zip.}



\paragraph{Exercice 4 Tutorial 2}
\href{http://mery54.github.io/teaching/mosos/models/fx4-tut2.zip}{Archive 
  Rodin   fx4-tut2.zip.}




\paragraph{Exercice 5 Tutorial 2}
\href{http://mery54.github.io/teaching/mosos/models/fx5-tut2.zip}{Archive 
  Rodin   fx5-tut2.zip.}



\paragraph{Exercice 6 Tutorial 2}
\href{http://mery54.github.io/teaching/mosos/models/fx6-tut2.zip}{Archive 
  Rodin   fx6-tut2.zip.}



 \subsubsection{Event-B Archives for   the tutorial 3}
\label{sec:event-b-archives}

\paragraph{Exercice 1 Tutorial 3}

Nous donnons deux solutions possibles: l'une avec ok et l'autre sans ok.

\href{http://mery54.github.io/teaching/mosos/models/qqx1-tut3.zip}{Archive 
  Rodin   ggx1-tut3.zip.}




\paragraph{Exercice 2 Tutorial 3}

\href{http://mery54.github.io/teaching/mosos/models/ggx2-tut3.zip}{Archive 
  Rodin   ggx2-tut3.zip.}





%projet

  \subsection{Project}
\label{sec:project}

The assessment  of students is based on two works:
\begin{itemize}
\item A written  exam   1h30
\item A project which is stated in the following document
  \href{http://mery54.github.io/teaching/mosos/lecturesnotes/master-projets2425.pdf}{Projet
    Master } 
\end{itemize}

The deadline for the project is  February 15, 2025 and the presentation
will be  organised later in February 2025. 



\subsection{Past exams  of  the  Course MsC in Computer Science: Modelling and verifying software-based systems for Master in Comp}
\label{sec:past-exams-course}


\href{http://mery54.github.io/teaching/mosos/lecturesnotes/annales-master2425.zip}{Archiuves
sous forme d'une archive ZIP}


\hrulefill




\section{Course DISCONT: Modelling hybrid systems}
\label{sec:course-discont}


\subsection{Documentation}
\label{sec:documentation}



The report entitled \href{https://inria.hal.science/hal-02895528v1
}{A Refinement Strategy for Hybrid System Design with Safety
  Constraints} is giving details on the method Event-B in hybrid
strategy.


The report entitled 
\href{https://hal.science/hal-04189025v2
}{From System Events to Software Operations for Refinement-based
  Modeling of Hybrid Systems}  studies the integreation of both
approaches  \textit{system} and \textit{software}.


\href{https://hal.science/hal-04152829v1
  }{A Static Checker for Reference Tracking Systems via Laplace Transform and Transfer Functions.}





\subsection{Slides for the course 2020-2021}

\begin{itemize}
\item[]  
  \href{http://mery54.github.io/teaching/mosos/lecturesnotes/2021-01-MLS-00.pdf}{Lecture
    1 }
\item[]  
  \href{http://mery54.github.io/teaching/mosos/lecturesnotes/2021-01-MLS-01.pdf}{Lecture
    2 }
\end{itemize}



\subsection{Slides for the course 2021-2022}

\begin{itemize}
\item[]  
  \href{http://mery54.github.io/teaching/mosos/lecturesnotes/2021-22-MLS-00.pdf}{Lecture
    1 }
\item[]  
  \href{http://mery54.github.io/teaching/mosos/lecturesnotes/2021-22-MLS-01.pdf}{Lecture
    2 }
  \item[]  
  \href{http://mery54.github.io/teaching/mosos/lecturesnotes/2021-22-MLS-02.pdf}{Lecture
    3 }
  \item[]  
  \href{http://mery54.github.io/teaching/mosos/lecturesnotes/2021-22-MLS-03.pdf}{Lecture 4 }
\end{itemize}


\bibliographystyle{plain}

\bibliography{references}
\end{document}


