\documentclass[ 12pt]{article}

\usepackage{times,vmargin,url}
\setmarginsrb{3cm}{1cm}{2cm}{1cm}{1cm}{1cm}{1cm}{1cm}
%\setpapersize{A4}


\usepackage{color}
 \pagecolor{yellow}
%\pagecolor{cyan}

\title{Modelling Software-based Systems}


\newcommand{\keywords}[1]{\textbf{Keywords:} #1} %\\ \clearpage %\tableofcontents\clearpage}
%

\author{Dominique M\'ery\\
LORIA \& Telecom Nancy\\ Universit\'e de Lorraine\\ \url{dominique.mery@loria.fr}}

\date{\today}

\usepackage{xcolor}
%\usepackage{b2latex}

%\IEEEoverridecommandlockouts
% The preceding line is only needed to identify funding in the first footnote. If that is unneeded, please comment it out.
\usepackage{cite,time}
\usepackage{amsmath,amssymb,amsfonts}
\usepackage{algorithmic}
\usepackage{graphicx}
\usepackage{textcomp}
\usepackage{xcolor,time,url,nbcode,version}
\usepackage{wrapfig}
%\usepackage{b2latex}
\usepackage{caption}
\usepackage{bm}
\usepackage{float}
\usepackage{enumitem}
\usepackage{listings}
\usepackage{bussproofs}
\usepackage{wrapfig,setspace}
%\usepackage{lstcoq}
\usepackage{xspace}
\usepackage{hyperref}

\def\BibTeX{{\rm B\kern-.05em{\sc i\kern-.025em b}\kern-.08em
    T\kern-.1667em\lower.7ex\hbox{E}\kern-.125emX}}
\pagestyle{plain}
\pagenumbering{arabic}
\let\proof\relax
\let\endproof\relax
\let\example\relax
\let\endexample\relax



\usepackage{mathpartir}
\usepackage{amssymb,amsmath,amsthm}
\usepackage{rotating}
\usepackage{blox}
\usepackage{graphicx} 
\usepackage{eventblst}
\usepackage{cite}
\usepackage{dafny}
 

% \usepackage{amssymb,amsmath} problematic with acm, use \usepackage{newtxtext,newtxmath}
% \usepackage{newtxtext,newtxmath}



%%%%%%%%%%%%%%%%%%%%%%%%%%%%%%%%%%%%%%%%%%%%%%%%%%%%%%%%%%%%%%%%%%%%%%%%
% Macros for proof-reading
%%%%%%%%%%%%%%%%%%%%%%%%%%%%%%%%%%%%%%%%%%%%%%%%%%%%%%%%%%%%%%%%%%%%%%%%


\usepackage{xcolor}
\usepackage{ifthen}
\newboolean{showcomments}
\setboolean{showcomments}{true} % toggle to show or hide comments

\ifthenelse{\boolean{showcomments}} { 
    \usepackage{outlines}
    \let\oldoutline\outline
    \def\outline{\oldoutline\color{blue}}

    \usepackage[normalem]{ulem}
    \def \ifempty#1{\def\temp{#1} \ifx\temp\empty }
    \newcommand{\xnote}[3][]{\textcolor{red}{#2}$\rightarrow$ \textcolor{blue}{\textbf{#1} #3}}
    \newcommand{\xtodo}[2][]{\xnote[#1]{}{#2}}

    \newcommand{\xreplace}[3][]{\ifempty {#3} \textcolor{red}{\textbf{#1} \sout{#2}} \else \textcolor{red}{\sout{#2}} $\rightarrow$ \textcolor{blue}{\textbf{#1} "#3"} \fi}
    \newcommand{\xdelete}[2][]{\xreplace[#1]{#2}{}}
    \newcommand{\xadd}[2][]{\xreplace[#1]{}{#2}}
} {
    \newcommand{\xnote}[3][]{}
    \newcommand{\xtodo}[2][]{}
    \newcommand{\xreplace}[3][]{#3}
    \newcommand{\xdelete}[2][]{}
    \newcommand{\xadd}[2][]{#2}
    \def\outline{}
    \def\1{}
    \def\2{}
    \def\3{}
    \def\4{}
}
% % % % % % % % % % % % % % % % % % % % % % % % %

\newcommand{\dnote}[2]{\xnote[DM]{#1}{#2}}
\newcommand{\dtodo}[1]{\xtodo[DM]{#1}}
\newcommand{\dreplace}[2]{\xreplace[DM]{#1}{#2}}
\newcommand{\ddelete}[1]{\xdelete[DM]{#1}}
\newcommand{\dadd}[1]{\xadd[DM]{#1}}


\newcommand{\znote}[2]{\xnote[ZC]{#1}{#2}}
\newcommand{\ztodo}[1]{\xtodo[ZC]{#1}}
\newcommand{\zreplace}[2]{\xreplace[ZC]{#1}{#2}}
\newcommand{\zdelete}[1]{\xdelete[ZC]{#1}}
\newcommand{\zadd}[1]{\xadd[ZC]{#1}}

% % % % % % % % % % % % % % % % % % % % % % % % %

\newcommand{\PB}{{\sc\bf  PB}\;}
\newcommand{\MO}{{\sc\bf  MO}\;}
\newcommand{\GC}{{\sc\bf  GC}\;}
\newcommand{\HP}{{\sc\bf  HP}\;}
\newcommand{\dL}{d${\cal  L}$\;}

\newcommand*\choice[0]{[\!]}
\newcommand{\set}[1]{\{#1\}}
\newcommand{\aloop}[1]{\textbf{do}\  #1  \ \textbf{od}}

\makeatletter
\def\footnoterule{\relax%
  \kern-5pt
  \hbox to \columnwidth{\hfill\vrule width 0.9\columnwidth height 0.4pt\hfill}
  \kern4.6pt}
\makeatother
%ddd -aal\renewcommand{\baselinestretch}{0.90}
\newtheorem{myexample}{Example}[section]
\newtheorem{mytheorem}{Property}
%\newtheorem{definition}{Definition}
%\includeversion{check}
% \excludeversion{check}
% \excludeversion{long}
%%%%%%%%%%%%%%%%%%%%%%%%%%%%%%%%%%%%%%%%%%%%%%%%
%%%%%%%%%%%%%%%%%%%%%%%%%%%%%%%%%%%%%%%%%%%%%%%%
\newtheorem{exemple}{Example}
%\newtheorem{theorem}{Theorem}
%\newcommand{\white}[1]{\textcolor{white}{#1}}

%\newtheorem{comm}{Teaching Point}
%\newenvironment{comm}{\iffalse}{\fi}
%\excludeversion{comm}

\input eb2latex

\pagestyle{plain}
%%%%%%%%%%%%%%%%%%%%%%%%%%%%%%%%%%%%%%%%%%%%%%%%
%%%%%%%%%%%%%%%%%%%%%%%%%%%%%%%%%%%%%%%%%%%%%%%%
\begin{document}
\newcounter{ex}  \setcounter{ex}{1}
\maketitle

\begin{abstract}

  The Event-B  method is based on a modelling
 language used to describe state-based models and  safety
 properties of those state-based models.  The originality of  Event-B
 lies in its ability  to
 enable incremental and proof-based modelling of \textit{reactive
   systems}. The  Event-B   language contains both set notations and a
 first-order predicate calculus; it offers the possibility of defining
 models of reactive systems called machines and contexts and includes
  the refinement relationship that allows us to follow an incremental
  development methodology.  
  
\end{abstract}

This site contains resources relating to the use of the Event-B
language and the Rodin platform to verify contracts for a small
sequential programming language.  We give the Rodin models and a
description of these models in the form of a text \LaTeX.


\section{Documentation}




\begin{itemize}
\item  The chapter \cite{losl-b} (from \cite{losl-eatcs}.)  authored by Dominique Cansell and
  Dominique M{\'e}ry, and entitled  \textit{The Event-B Modelling
  Method: Concepts and Case        Studies} has benne published from
lectures notes given in a Summer Schooll and you can use it for
getting details from  Event-B   see  the  \href{http://mery54.github.io/teaching/mosos/lecturesnotes/BasicEventB.pdf}{following
  link}.
  
\end{itemize}
\section{Course MCFSI  at Telecom Nancy}
\label{sec:course-mcfsi-at}




\begin{itemize}
\item  Lectures Notes   \textit{The Modelling Language} at the
  \href{http://mery54.github.io/teaching/mosos/lecturesnotes/main-Poly1.pdf}{following
    link}.
\item Slides of the course
  \begin{itemize}
\item Lecture 1
  \href{http://mery54.github.io/teaching/mosos/lecturesnotes/mcfsi-lect1.pdf}{The 
    Modelling Language Event-B}.
  
  \item Lecture 2
    \href{http://mery54.github.io/teaching/mosos/lecturesnotes/mcfsi-po.pdf}{Proof
      Obligations}.
     \item Lecture 3 
    \href{http://mery54.github.io/teaching/mosos/lecturesnotes/mcfsi-lect2.pdf}{Correctness by Construction with the  
      Modelling Language Event-B using the Refinement}.
         \item Lecture 4
    \href{http://mery54.github.io/teaching/mosos/lecturesnotes/mcfsi-lect3.pdf}{Access
      Control }
    
      \end{itemize}
\item Tutorials
  \begin{itemize}
  \item   Tutorial 1
    \href{http://mery54.github.io/teaching/mosos/lecturesnotes/tutorial1.pdf}{Using
      the Event-B modelling language}.
  \end{itemize}
\item  The Event-B models are at the
  \href{http://mery54.github.io/teaching/models/}{link for
    \textsf{Event-B} }.
  
\item Slides for the tutorial  at
  \href{http://mery54.github.io/teaching/mosos/lecturesnotes/mcfsi-maynooth-tut.pdf}{Tutorial
  May 2, 2025}.
  \end{itemize}



\bibliographystyle{plain}

\bibliography{references}
\end{document}


