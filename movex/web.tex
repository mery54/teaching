\documentclass[ 12pt]{article}
%\dose cctioncumentclass[ 12pt]{book}

\usepackage{times,vmargin,url}
%\setmarginsrb{3cm}{1cm}{2cm}{1cm}{1cm}{1cm}{1cm}{1cm}
\setpapersize{A4}


\usepackage{color,time}
%\pagecolor{yellow}
%\pagecolor{cyan}
\pagecolor{white}
\title{Modelling,  verification   and experimentation  for 
  software-based systems (MOVEX)}


\newcommand{\keywords}[1]{\textbf{Keywords:} #1} %\\ \clearpage %\tableofcontents\clearpage}
%

\author{Dominique M\'ery\\
LORIA \& Telecom Nancy\\ Universit\'e de Lorraine\\
\url{https://members.loria.fr/Mery}\\ \url{dominique.mery@loria.fr}}

\date{\today}

\usepackage{xcolor}
%\usepackage{b2latex}

%\IEEEoverridecommandlockouts
% The preceding line is only needed to identify funding in the first footnote. If that is unneeded, please comment it out.
\usepackage{cite,time}
\usepackage{amsmath,amssymb,amsfonts}
\usepackage{algorithmic}
\usepackage{graphicx}
\usepackage{textcomp}
\usepackage{xcolor,time,url,nbcode,version}
\usepackage{wrapfig}
%\usepackage{b2latex}
\usepackage{caption}
\usepackage{bm}
\usepackage{float}
\usepackage{enumitem}
\usepackage{listings}
\usepackage{bussproofs}
\usepackage{wrapfig,setspace}
%\usepackage{lstcoq}
\usepackage{xspace}
\usepackage{hyperref}

\def\BibTeX{{\rm B\kern-.05em{\sc i\kern-.025em b}\kern-.08em
    T\kern-.1667em\lower.7ex\hbox{E}\kern-.125emX}}
\pagestyle{plain}
\pagenumbering{arabic}
\let\proof\relax
\let\endproof\relax
\let\example\relax
\let\endexample\relax



\usepackage{mathpartir}
\usepackage{amssymb,amsmath,amsthm}
\usepackage{rotating}
\usepackage{blox}
\usepackage{graphicx} 
\usepackage{eventblst}
\usepackage{cite}
\usepackage{dafny}
 

% \usepackage{amssymb,amsmath} problematic with acm, use \usepackage{newtxtext,newtxmath}
% \usepackage{newtxtext,newtxmath}



%%%%%%%%%%%%%%%%%%%%%%%%%%%%%%%%%%%%%%%%%%%%%%%%%%%%%%%%%%%%%%%%%%%%%%%%
% Macros for proof-reading
%%%%%%%%%%%%%%%%%%%%%%%%%%%%%%%%%%%%%%%%%%%%%%%%%%%%%%%%%%%%%%%%%%%%%%%%


\usepackage{xcolor}
\usepackage{ifthen}
\newboolean{showcomments}
\setboolean{showcomments}{true} % toggle to show or hide comments

\ifthenelse{\boolean{showcomments}} { 
    \usepackage{outlines}
    \let\oldoutline\outline
    \def\outline{\oldoutline\color{blue}}

    \usepackage[normalem]{ulem}
    \def \ifempty#1{\def\temp{#1} \ifx\temp\empty }
    \newcommand{\xnote}[3][]{\textcolor{red}{#2}$\rightarrow$ \textcolor{blue}{\textbf{#1} #3}}
    \newcommand{\xtodo}[2][]{\xnote[#1]{}{#2}}

    \newcommand{\xreplace}[3][]{\ifempty {#3} \textcolor{red}{\textbf{#1} \sout{#2}} \else \textcolor{red}{\sout{#2}} $\rightarrow$ \textcolor{blue}{\textbf{#1} "#3"} \fi}
    \newcommand{\xdelete}[2][]{\xreplace[#1]{#2}{}}
    \newcommand{\xadd}[2][]{\xreplace[#1]{}{#2}}
} {
    \newcommand{\xnote}[3][]{}
    \newcommand{\xtodo}[2][]{}
    \newcommand{\xreplace}[3][]{#3}
    \newcommand{\xdelete}[2][]{}
    \newcommand{\xadd}[2][]{#2}
    \def\outline{}
    \def\1{}
    \def\2{}
    \def\3{}
    \def\4{}
}
% % % % % % % % % % % % % % % % % % % % % % % % %

\newcommand{\dnote}[2]{\xnote[DM]{#1}{#2}}
\newcommand{\dtodo}[1]{\xtodo[DM]{#1}}
\newcommand{\dreplace}[2]{\xreplace[DM]{#1}{#2}}
\newcommand{\ddelete}[1]{\xdelete[DM]{#1}}
\newcommand{\dadd}[1]{\xadd[DM]{#1}}


\newcommand{\znote}[2]{\xnote[ZC]{#1}{#2}}
\newcommand{\ztodo}[1]{\xtodo[ZC]{#1}}
\newcommand{\zreplace}[2]{\xreplace[ZC]{#1}{#2}}
\newcommand{\zdelete}[1]{\xdelete[ZC]{#1}}
\newcommand{\zadd}[1]{\xadd[ZC]{#1}}

% % % % % % % % % % % % % % % % % % % % % % % % %

\newcommand{\PB}{{\sc\bf  PB}\;}
\newcommand{\MO}{{\sc\bf  MO}\;}
\newcommand{\GC}{{\sc\bf  GC}\;}
\newcommand{\HP}{{\sc\bf  HP}\;}
\newcommand{\dL}{d${\cal  L}$\;}

\newcommand*\choice[0]{[\!]}
\newcommand{\set}[1]{\{#1\}}
\newcommand{\aloop}[1]{\textbf{do}\  #1  \ \textbf{od}}

\makeatletter
\def\footnoterule{\relax%
  \kern-5pt
  \hbox to \columnwidth{\hfill\vrule width 0.9\columnwidth height 0.4pt\hfill}
  \kern4.6pt}
\makeatother
%ddd -aal\renewcommand{\baselinestretch}{0.90}
\newtheorem{myexample}{Example}[section]
\newtheorem{mytheorem}{Property}
%\newtheorem{definition}{Definition}
%\includeversion{check}
% \excludeversion{check}
% \excludeversion{long}
%%%%%%%%%%%%%%%%%%%%%%%%%%%%%%%%%%%%%%%%%%%%%%%%
%%%%%%%%%%%%%%%%%%%%%%%%%%%%%%%%%%%%%%%%%%%%%%%%
\newtheorem{exemple}{Example}
%\newtheorem{theorem}{Theorem}
%\newcommand{\white}[1]{\textcolor{white}{#1}}

%\newtheorem{comm}{Teaching Point}
%\newenvironment{comm}{\iffalse}{\fi}
%\excludeversion{comm}

\input eb2latex

\pagestyle{plain}
%%%%%%%%%%%%%%%%%%%%%%%%%%%%%%%%%%%%%%%%%%%%%%%%
%%%%%%%%%%%%%%%%%%%%%%%%%%%%%%%%%%%%%%%%%%%%%%%%
\begin{document}
\newcounter{ex}  \setcounter{ex}{1}
\maketitle
\begin{abstract}
This repository contains course notes, exercises, models and projects
from two  courses given as part of master's level training on
modelling an verifying software-based systems. It provides access to
resources in the form of pdf files, TLA files, ACSL files  or Rodin
files. Moreover, it aims to prepare  students of the fourth year of
University to apply modelling techniques  for software-based systems.
It is divided into  two  main  parts:
\begin{itemize}
\item Part 1  \textsf{MALG} is shared by students in software engineering and in
 CPS engineering; the course is oganised  in 6 weeks (6 lectures x 
 2h00 ) (6 tutorials x 2 h 00). Topics are  transition systems,
 invariance, safety, fixed-point theory, induction principles,
 Floyd/Hoare proof systems, 
\item Part 2 is divided into  two distinct streams:
  \begin{itemize}
  \item \textsf{MOVEX-SE} is the course is oganised  in 6 weeks (6 lectures x 
 2h00 ) (6 tutorials x 2 h 00)
  \item  \textsf{MOVEX-CPS} the course is oganised  in 6 weeks (6 lectures x 
 2h00 ) (6 tutorials x 2 h 00)
  \end{itemize}
\end{itemize}

The table
of contents shows the  summary of  two main courses (at Université de 
 Lorraine/University of Lorraine)  based on our
 experiment using the modelling languages as TLA,   Event-B  and ACSL
 
\begin{itemize}
\item The first course \textbf{MALG} is part of the curriculum of the
  second  year
  students of Telecom Nancy  who are  focusing on software
  engineering.
\item  The second course  \textbf{MOVEX}  is part of the curriculum of the
  second  year
  students of Telecom Nancy  who are  focusing on embedded systems, as
  well as students of seconf year of ENSEM. 
\end{itemize}

  
\end{abstract}

\tableofcontents


\section{Documentation and Tools}

\begin{itemize}

    
\item[]  The  TLA+ ToolBox  platform is available at the 
  \href{https://lamport.azurewebsites.net/tla/toolbox.html}{following 
    link}. 
\item[]  The Rodin platform is available at the 
  \href{https://www.event-b.org/install.html}{following 
    link}.

  
  \item[]  The Prob   platform is available at the 
  \href{https://prob.hhu.de}{following 
    link}. 

  
  \item[]  The Frama-c platform is available at the 
  \href{https://www.frama-c.com}{following 
    link}.

  \item[]  The Synchrone Reactive Toolbox for LUSTRE is   available at the 
  \href{https://www-verimag.imag.fr/Outils-SynchronesNEW.html?lang=en}{following 
    link}.

  \item[]  The Kind 2  platform is available at the 
  \href{ https://kind2-mc.github.io/kind2/}{following 
    link}.

  
\end{itemize}




\section{Course MALG1/MOVEX1 at Telecom Nancy}
\label{sec:course-mcfsi-at}


\subsection{Slides for the course MALG1/MOVEX1}
\label{sec:slides}


\subsubsection{Lecture 0 {Overview of the course }}
  
  \href{http://mery54.github.io/teaching/movex/lecturesnotes/movexlecture0.pdf}{Overview of the course }


\subsubsection{Lecture 1 {Mod\'elisation, 
    sp\'ecification et v\'erification}  (I) }
  
\begin{itemize}
\item
  \href{http://mery54.github.io/teaching/movex/lecturesnotes/movexlecture1.pdf}{Mod\'elisation,
    sp\'ecification et v\'erification (I)}
\item
  \href{http://mery54.github.io/teaching/movex/tlafolder/access_control.tla}{The
    MODULE access\_control.tla}.  
 
\end{itemize}


  
\subsubsection{Lecture 2 {Mod\'elisation,      sp\'ecification et v\'erification} (II)}
  
  \href{http://mery54.github.io/teaching/movex/lecturesnotes/movexlecture2.pdf}{Mod\'elisation,     sp\'ecification et v\'erification (II)}


  \subsubsection{Lecture 3 {Mod\'elisation,      sp\'ecification et v\'erification} (III)}
  
  \href{http://mery54.github.io/teaching/movex/lecturesnotes/movexlecture3.pdf}{Mod\'elisation,     sp\'ecification et v\'erification (III)}


  
  \subsubsection{Lecture 4 {Vérification  mécanisée de contrats} (I)}
  
\href{http://mery54.github.io/teaching/movex/lecturesnotes/movexlecture4.pdf}{{Vérification  mécanisée de contrats} (I)}



  \subsubsection{Lecture 5 {Vérification  mécanisée de contrats} (II)}
  
\href{http://mery54.github.io/teaching/movex/lecturesnotes/movexlecture5.pdf}{{Vérification  mécanisée de contrats} (II)}



  

  \subsection{Lectures Notes}

  \begin{itemize}
  \item[]   Notes sur la logique
    \href{http://mery54.github.io/teaching/movex/lecturesnotes/preprint-chapterlogique.pdf}{Notes sur la logique}.

     \item[]   Notes sur la vérification
    \href{http://mery54.github.io/teaching/movex/lecturesnotes/preprint-chapterverification.pdf}{Notes
      sur la vérification}.

     \item[]   Notes sur la preuve
    \href{http://mery54.github.io/teaching/movex/lecturesnotes/preprint-chapterprouver.pdf}{Notes
      sur la preuve}.

  \end{itemize}


  

\subsection{Tutorials}



\begin{itemize}
  
\item[]  Notes on a tutorial using proofs 
    \href{http://mery54.github.io/teaching/movex/lecturesnotes/notestuto3.pdf}{Tutorial
      Notes for Proofs and Verifications Conditions of Contract}.
  \item[]   Serie  1
    \href{http://mery54.github.io/teaching/movex/lecturesnotes/movexserie1.pdf}{Tutorials 
      serie 1}.

 %         \href{http://mery54.github.io/teaching/movex/lecturesnotes/correctionmovexserie1.pdf}{Tutorials 
 %     serie 1 with solutions}. 

    
        \href{http://mery54.github.io/teaching/movex/models/serie1.zip}{TLA 
           solutions for serie 1
         }.
\item[]   Serie  2
    \href{http://mery54.github.io/teaching/movex/lecturesnotes/movexserie2.pdf}{Tutorials 
      serie 2}.

%      \href{http://mery54.github.io/teaching/movex/lecturesnotes/correctionmovexserie2.pdf}{Tutorials 
%      serie 2 with solutions}. 

           \href{http://mery54.github.io/teaching/movex/models/serie2.zip}{TLA 
             solutions for serie 2
             }.
  \end{itemize}




  
  
  
  \subsection{Assessment}
\label{sec:project}

The assessment  of students is based on three works:
\begin{itemize}
\item Two  written  exams: E1 and E2
\item  A practical exam: TP
 \end{itemize}



\bibliographystyle{plain}
\bibliography{references}
\end{document}




      
